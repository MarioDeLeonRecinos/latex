\documentclass[11pt]{article}

\usepackage[margin=1in]{geometry}
\usepackage{amsfonts, amsmath, amssymb}
\usepackage[none]{hyphenat}
\usepackage{fancyhdr}
\usepackage{graphicx}
\usepackage{float}
\usepackage{mathtools}
\usepackage{enumitem}
\usepackage{comment}
\usepackage[nottoc, notlot, notlof]{tocbibind}

\pagestyle{fancy}
\fancyhead{}
\fancyfoot{}
\fancyhead[L]{\slshape\MakeUppercase{Método de Newton para sistemas no lineales}}
%\fancyhead[R]{\slshape{Student Name}}
\fancyfoot[C]{\thepage}
%\renewcommand{\headrulewidth}{0pt}
\renewcommand{\footrulewidth}{0pt}

\DeclarePairedDelimiter\abs{\lvert}{\rvert}%
\DeclarePairedDelimiter\norm{\lVert}{\rVert}%

% Swap the definition of \abs* and \norm*, so that \abs
% and \norm resizes the size of the brackets, and the 
% starred version does not.
\makeatletter
\let\oldabs\abs
\def\abs{\@ifstar{\oldabs}{\oldabs*}}
%
\let\oldnorm\norm
\def\norm{\@ifstar{\oldnorm}{\oldnorm*}}
\makeatother

\parindent 0ex
%\setlength{\parindent}{4em}
%\setlength{\parskip}{1em}
\renewcommand{\baselinestretch}{1.5}

\begin{document}
\begin{titlepage}

\begin{center}
%\vspace*{1cm}
\Large{\textbf{Analisis Numerico}}\\
\Large{\textbf{Proyecto Final}}\\
\vfill
\line(1,0){500}\\[1mm]

\huge{\textbf{Método de Newton para sistemas no lineales}}\\[3mm]
\Large{\textbf{- Implementacion y teoria del metodo -}}\\[1mm]

\line(1,0){500}\\
\vfill
00076015 Carlos Javier Burgos Martinez\\
00006715 David Bejamín Ayala Giralt\\
00388913 Diego José Eguizabal Liu\\
00058615 Karla Esperanza López Méndez\\
00353715 Mario Cecilio De Leon Recinos\\
00004315 Rafael Enrique Cruz Aparicio \\
00088116 Yury Alejandro Rivera Quintanilla\\
\vfill
\today\\

\end{center}

\end{titlepage}

\tableofcontents
\thispagestyle{empty}
\clearpage

\setcounter{page}{1}

\section{Definiciones}

El problema computacional que se tiene a resolver es solucion de ecuaciones no lineales, el cual sera resuelto con el metodo de newton para ecuaciones no lineales. Este metodo tiene un procedimiento algoritmico para efectuar la transformacion en una situacion mas general.\\

Para construir dicho algoritmo que lleve a una solucion del metodo de punto-fijo en un caso unidimensional, obtuvimos una funcion $\phi$ con las propiedades $$g(x)=x-\phi(x)f(x)$$ da una convergencia cuadratica en el punto fijo $p$ de la funcion $g$.De esta condicion el metodo de Newton evoluciono al escoger $\phi(x)=1/f'(x)$ asumiendo que $f'(x)\neq0$.\\

Un enfoque similar in el caso $n$-dimensional implica una matriz

\begin{equation}
A=
\begin{bmatrix}
    a_{11}       & a_{12} & \dots & a_{1n} \\
    a_{21}       & a_{22} & \dots & a_{2n} \\
    \vdots & \vdots & \ddots  & \vdots\\
    a_{n1}       & a_{n2} & \dots & a_{nn}
    \label{matrix:A}    
\end{bmatrix}
\end{equation}


donde cada una de las entradas $a_{ij}(x)$ es una funcion de $\mathbb{R}^n$ a $\mathbb{R}$ . Esto requiere que $A(x)$ sea encontrado para que $$\mathbf{G(x)=x-A(x)^{-1}F(x)}$$ genere una convergencia cuadratica para la solucion $\mathbf{F(x)=0}$, asumiendo que $A(x)$ es no singular en el punto fijo $\mathbf{p}$ de $\mathbf{G}$.\\

Suponemos que $p$ es una solucion de $G(x)=x$. Si existe un numero $\delta>0$ con la propiedad que

\begin{enumerate}[label=(\roman*)]

	\item $\partial g_i / \partial x_j$ sea continua en $N_\delta=x|\quad \norm{x-p}$ para toda $i=1,2,...,n$ y toda $j=1,2,///,n;$
	\item $\partial^2 g_i(x) / \partial x_j\partial x_k$ sea continua y $\abs{ \partial^2 g_i(x)/(\partial x_j\partial x_k)}\leq M$ para alguna constante M siempre que $x\in N_\delta$ para toda $i=1,2,...,n, j=1,2,...,n$ y toda $k=1,2,...,n;$ 
	\item $\partial g_i(p) /\partial x_k =0$ para toda $i=1,2,...,n$ y toda $k=1,2,...,n.$

\end{enumerate}

Entonces existe un numero $\hat{\delta}\leq \delta$ tal que la sucesion generada por $x^k=G(x^{k-1})$
converge cuadraticamente a $p$ para cualquier eleccion de $x^0$ a condicion de que $\quad \norm{x^0-p}<\hat{\delta}$
\begin{center}
$\quad \norm{x^k-p}_\infty \leq \frac{n^2M}{2}\quad \norm{x^{k-1}-p}_\infty^2$ para toda k$\geq 1$
\end{center}
Para utilizar el teorema supongamos que $A(x)$ es una matriz de $n X n$ de funcio­nes de $\mathbb{R}^n$ a $\mathbb{R}$ en la forma de la ecuacion \ref{matrix:A}, cuyos elementos específicos se escoge­rán más adelante. Supongamos además que $A(x)$ es no singular cerca de una solución $p$ de $F(x)=O$ , y denotemos con $b_{ij}(x)$ el elemento de $A(x)^{-1}$ en el i-ésimo renglón y en la j-ésima columna.

\begin{table}[H]
	\centering
		\begin{tabular}{|c|c|c|c|}\hline
		$x$ &0&1&2\\ \hline
		$f(x)$ &3&6&9\\ \hline		
		\end{tabular}
	\caption{Caption goes here}
	\label{tab:data1}
\end{table}


\section{Desarrollo del programa}

\subsection{Algoritmo}
sujeto a cambios
Una vez el método de Newton ha sido adaptado para trabajar con $\mathbb{n}$ funciones y $\mathbb{n}$ variables de forma tal que el  
sujeto a cambios
El algoritmo empleado para resolver sistemas de ecuaciones no lineales con el método de Newton aun posee todas sus cualidades es decir la velocidad de convergencia del método no se vio afectada como resultado de adaptarlo al caso donde se ncesita trabajar con un espacio vectorial de funciones que mapeen $\mathbb{R}^3$ a $\mathbb{R}^3$


\subsection{Pseudocódigo}
El Pseudocódigo para el proyecto se divide en 3 bloques principales de codigo:
\begin{enumerate}
  \item Configuración de parametros
  \item Bucle
  \item Salida
\end{enumerate}

\subsubsection{Configuración de parametros}
Durante la configuración de parametros el programa necesita recibir como entrada varias cosas, entre ellas estan el número  $n$ de incognitas y de funciones que se desea calcular, luego la precision deseada que luego servira para saber en que momento detener el programa. Finalmente se necesitaran las ecuaciones que deben ser solucionadas.

\subsubsection{Bucle}
Esta es la parte computacionalmente mas exigente del programa, en esta parte se necesitan calcular los siguientes valores
\begin{itemize}
  \item Matris Jacobiana de la iteración
  \item El vector $F(\bar{X})$
  \item Salida
\end{itemize}
\clearpage
%\thispagestyle{empty}
\bibliographystyle{plain}
\nocite{*}
\bibliography{bibliografia}

\end{document}